
\newpage
\anonsection{Введение}

С момента зарождения компьютерной графики одним из самых популярных объектов её исследований оставалось человеческое лицо. Как только появились методы генерации синтетических изображений, были предприняты попытки нарисовать человеческое лицо с помощью компьютера, как можно достовернее воспроизвести мимику и передать эмоции, а впоследствии --- синтезировать на компьютере реалистичные лица.

Методы <<чистой>> компьютерной графики при этом уделяют особое внимание передаче физических свойств поверхности, таких как освещённость, отражательная способность, подповерхностное рассеивание света в коже.

В то же время методы компьютерного зрения сосредоточены на анализе изображений лиц реальных людей, вычленении из них определённых наборов локальных и глобальных особенностей и комбинации их с целью получить результат, похожий на изображение из обучающей выборки. Физическим свойствам поверхности при этом уделяется второстепенное внимание.

Параметрические модели лиц, используемые в таких методах, обычно принимают в качестве параметров цвет и текстуру кожи, пропорции частей лица, описание выражения лица, словом, те параметры, которые очевидным образом задают правила для синтеза изображения. Эти параметры для реального лица возможно подобрать множеством способов, например, решая оптимизационную задачу. Однако, параметрами для синтеза изображения лица могут служить и более сложные характеристики, напрямую не связанные с внешним видом лица, такие как настроение, возраст или пол. Более того, становится возможным методами машинного обучения провести анализ и установить зависимость между <<простыми>> параметрами лица и <<сложными>>, например, можно получить классификатор, который по найденным <<простым>> параметрам лица распознаёт настроение, пол или возраст.

Отдельную задачу представляет собой определение возраста и связанная с ней задача изменения возраста по исходной фотографии. Задача определения возраста по фотографии помимо того, что является весьма сложной и интересной исследовательской задачей, имеет множество практических применений. Например, создание систем, автоматически подстраивающихся под пользователя с учётом его возраста, анализ аудитории и маркетинговые исследования, системы поиска изображений и поиска людей по изображениям. Задача изменения возраста лица по фотографии включает в себя задачу определения возраста, её решение даёт возможность состарить (или омолодить) лицо на изображении. Среди практических применений этой задачи можно выделить поиск пропавших людей (особенно детей), обновление фотографий в базах данных сотрудников, устойчивое к возрастным изменениям распознавание лиц и самое очевидное применение --- автоматическое омолаживание лица в эстетических целях.

Внешний вид лица в процессе старения зависит от многих факторов, таких как раса, пол, образ жизни, а также использование косметики, инъекций или хирургическое вмешательство. Из-за обилия этих факторов даже человек не всегда в состоянии определить возраст по фотографии достаточно точно. Создать алгоритм, способный решить такую задачу, ещё сложнее, поскольку мы сами до конца не осознаём, на какие признаки мы обращаем внимание, когда судим о возрасте, т.е. мы не знаем, как из параметров лица, которые мы можем детектировать, получить возраст. В таких случаях, когда неясно, как устроена функция от множества параметров, но известны её значения на некоторых входных данных, обычно используют машинное обучение.

Автоматическая имитация возраста предваряется анализом лиц различных возрастов и вычленением из них изменений, происходящих во время старения на разных возрастах. Для получения модифицированного изображения сначала производится предобработка исходного, затем --- автоматическое определение возраста, после чего к лицу применяются изменения, типичные для перехода от найденного возраста к целевому. 

Целью данной работы ставится анализ существующих методов автоматической оценки и автоматической имитации возраста с теоретической и практической точек зрения. Требуется изучить существующие на сегодня подходы к обеим задачам, определить, какие подзадачи необходимо для этого решить и реализовать наиболее важные из них. Поскольку задача имитации возраста включает в себя задачу определения возраста, было бы полезно рассмотреть задачу поиска возраста именно как часть системы по автоматической имитации возраста.